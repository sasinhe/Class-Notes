\documentclass[english, a4paper,12pt]{Iart}

\usepackage{amsmath}
\usepackage{amsfonts}
\usepackage{amsthm}
\usepackage{tcolorbox}
\swapnumbers % optional, of course
\newtheorem{thm}{Theorem}[section] % the main one
\newtheorem{lemma}[thm]{Lemma}
% other statement types

% for specifying a name
\theoremstyle{plain} % just in case the style had changed
\newcommand{\thistheoremname}{}
\newtheorem{genericthm}[thm]{\thistheoremname}
\newenvironment{namedthm}[1]
{\renewcommand{\thistheoremname}{#1}%
	\begin{genericthm}}
	{\end{genericthm}}

\newtheorem{definition}{Definition}

\newtheorem*{genericthm*}{\thistheoremname}
\newenvironment{namedthm*}[1]
{\renewcommand{\thistheoremname}{#1}%
	\begin{genericthm*}}	{\end{genericthm*}}


\theoremstyle{remark}
\newtheorem*{remark}{Remark}

\theoremstyle{definition}
\newtheorem{example}{Example}[section]

\newtheorem{theorem}{Theorem}

\begin{document}

\author{Samir Salmen}
\title{Complex Analysis}
\date{February 2021}
\newcommand{\C}{\mathbb{C}}
\newcommand{\R}{\mathbb{R}}



%Ideas:
%Add a "What will you learn in this chapter box"

\maketitle


\chapter{Lecture I: Introduction to Complex Analysis}


Complex Analysis is the natural analogy of Real Analysis to the field of real numbers.
like real numbers that are represented on a line, complex numbers (of the form $a+bi\,|\,a,b \in \mathbb{R}$ \t and  $i^2 = -1$) are represented on a plane.\\
Complex analysis is more close to geometric than algebraic thinking, this is justified by the concept that complex analysis can be defined as following:
\begin{definition}
    Complex analysis is the local relevant for the study of complex manifolds.
\end{definition}
This phrase contains much of the motivation about complex analysis. Lets develop this thought:
\begin{enumerate}
    \item Important in mathematics and physics.
        \begin{enumerate}
            \item String theory (with complex geometry)
            \item Analytic functions appear often on Fourier analysis, Harmonic analysis, number theory...  
        \end{enumerate}
        \begin{example}[Riemman zeta function]
            \begin{align*}
                \zeta(s) = \frac{1}{\Gamma(s)} \int^\infty_0 \frac{x^{s-1}}{e^x-1} \, dx. \, s\in \mathbb{c}
            \end{align*}
            Are all non trivial zeros of this function on the $Re[x] = \frac{1}{2}$ line?
        \end{example}
I\end{enumerate}
Lets define the goals of this course now:
\begin{itemize}
    \item Analytic functions and power series.
    \item the Riemann sphere and Mobius transformation.
    \item Complex integration.
    \item some "bonus topics".
\end{itemize}`


Now, lets extend the usual real concepts to the complex plane.

\begin{itemize}
    \item Trigonometric: $e^{ix} = \cos(x) + i\sin(x) \implies \cos(x) = \frac{\exp(ix) + \exp(-ix)}{2}$
    \item Differentiation: Different from Real Functions, there is only two classes of complex functions. $C^\infty$ and discontinuous ones.
    \item Integration: Like differentiation, integration is also different. When you integrate a function on the complex plane, the answer will depend on the path taken. However, \textbf{Cauchy Theorems} states that the integral of $f(x)$ \textit{almost} does not depend on th path taken (the set of functions that depend on it are called \textbf{non-holomorphic}).
    \item \textit{Difficult} real operations: We can learn to realize some difficult calculations more easily using complex analysis (like $\int^\infty_0 \frac{\sin(x)}{x} \, dx$).
    \item Analytic continuation:  A function that is \textit{complex differentiable} on $[a, b]$ has a \textit{unique} differentiable analytic continuation.
\end{itemize}

\section{Construction of $\mathbb{C}$}
\underline{Classical analysis}: Works over $(\mathbb{R}, +, \cdot)$ an \textbf{ordered field}, i.e.
\begin{itemize}
    \item $\forall x, y \in \mathbb{R}$, either $x < y,\,x > y,\,x=y$ holds true.\\
    \item $x, y > 0 \implies x+y > 0 \text{ and } xy > 0$.
\end{itemize}
However, $\mathbb{R}$ is \textbf{\textit{not} algebraic closed}. ($x^2 + 1$ has no roots /$\mathbb{R}$).\\

\underline{Question}: Can $(\mathbb{R}^2, +)$ (Abelian group) be made into a field? [what about $(\mathbb{R}^n, +)$?].\\

\begin{theorem}
    If this is possible, then $\exists z \in (\mathbb{R}^2, +, \cdot)$ such that $z^2 = -1$.
\end{theorem}
\begin{proof}
    Chose a basis $\{1, e\}\text{ of }\mathbb{R}^2$
    \begin{align*}
        z &= x \cdot 1 + y \cdot e \\
        z^2 &= x^2 \cdot 1 + 2xy \cdot e + y^2 \cdot e^2 \\
        e &= a\cdot 1 + b\cdot e, a,b \in \mathbb{R} \\
        z^2 &= (x + ay^2)\cdot 1 + (2xy + by^2)\cdot e \\
    \end{align*}
    We want to find $(x,y) \in \mathbb{R}$ s.t $z^2 = -1$. Then we must have $2x + by^2 = 0$. 
    So either $y = 0$ or $x = -\frac{b}{2}y$, lets discard the first one because it implies that $z \in \mathbb{R}$. Therefore, if we go after the second possibility a straight foward calculation gives:
    \begin{align*}
        z^2 &= ((-\frac{b}{2}y)^2 + ay^2)\cdot 1 + (0)\cdot e\\
        z^2 &= (a+\frac{b^2}{4})y \cdot 1\\
    .\end{align*}
    One can prove that $y = -\frac{1}{(a+\frac{b^2}{4})}$. Therefore $z^2 = -1$. This proof was first introduced by Gauss in his thesis, 1799.  
\end{proof}


\chapter{Lecture II: Complex Differentiation}
$\mathbb{C}$ is a metric space with $d(z,w) = |z-w|$ isometric to $\mathbb{R}^2$.\\
\underline{Triangle ineq}: $|z+w| \le |z| + |w|$.\\
\underline{Topology}: Open disks $\{z \in \mathbb{C}: |z-w| < R\} = D(w, R)$ (analogous with closed disks)\\
\begin{definition}[open sets in $\C$]
    $A \subset \C$ is open if $\forall a \in A$ has a disk $D(a, \epsilon) \le A$ for some $\epsilon > 0$.
\end{definition}

\section{Functions}
$f: \C \to \C$ can be viewed as $f: \R^2 \to \R^2$ because $\C \simeq R^2$.
Therefore, we have:
\begin{align*}
    f(z) = u(x,y) + iv(x,y)$
.\end{align*}
\subsection{Complex Differentiation}
\begin{align*}
    f: A &\to \C, a \subset \C\\
    \C field &\implies \text{ Can consider}\\
    \frac{f(z+h) - f(z)}{h} &\text{ is well defined}\\
.\end{align*}
Therefore, the derivative (if it exists) of $f(z)$ is defined as:
\begin{align*}
    f'(z) = \lim_{h\to 0} \frac{f(z+h)-f(z)}{h}
.\end{align*}
Note that means that this limit exists for every direction you take to approach the origin. This is indeed \textbf{a very strong condition}. This justifies the strange behavior of this kind of functions that we'll see later on.\\
\begin{defnition}
        A function is holomorphic on $A$ if $f$ is complex differentiable at each point $a \in A$.
\end{defnition}

\begin{tcolorbox}
    Note: All the usual rules of differentiation works on complex differentiation.\\
    \begin{itemize}
        \item (f+g)' = f' + g'\\
        \item (fg)' = f'g + fg'
        \item \ldots
    \end{itemize}
\end{tcolorbox}

Now, lets study a condition that implies holomorphism on functions.
\subsection{Cauchy-Riemann equations}
\begin{align*}
    z = x+iy; x,y \in \R
    f = y+iv; u,v: \R^2 \to R
.\end{align*}
We can approach the origin (in particular) along:
\begin{enumerate}
    \item $(x+k,y) \to (x,y)$
    \item $(x,y+ik) \to  (x,y)$
\end{enumerate}
\begin{enumerate}
    In \item $\frac{f(z+h)-f(z)}{h} = \frac{u(x+k,y)}{k} + i \frac{v(x+k,y)}{k}$\\
    $f$ holomorphic $\implies f'(z) = \left(\frac{\partial u}{\partial x} + i \frac{\partial u}{\partial x}\right)(x,y) $
    Doing the same for \item we get:
\end{enumerate}
\begin{align*}
    \frac{\partial u}{\partial x} = \frac{\partial v}{\partial y}\\
    \frac{\partial u}{\partial y} = - \frac{\partial v}{\partial x}  
\end{align*}
This is a necessary condition for holomorphism (called Cauchy-Riemann Equations). This isn't what we want exactly, but is a good progress. The sufficient condition will be described next.
\subsection{"Partial Converse"}
\begin{theorem}[Goursat]
    if $u,v: \R^2 \supset A \to \R$ satisfy the CR equations and $\frac{\partial u}{\partial x}$, $\frac{\partial u}{\partial y}$, $\frac{\partial v}{\partial x}$ and $\frac{\partial v}{\partial y}$ are continous.\\
    Then $ f := u +iv$ is holomorphic on $A$.
\end{theorem}

\subsection{Alternative way to think about holomorphic function}

\section{Power series}
\underline{Recall:} X metric space.\\
\underline{Def.}
\begin{enumerate}
    \item A sequence $x_{n} \in X$ is Cauchy $\iff \forall \epsilon > 0 \exists N \in
\end{enumerate}


\end{document}
