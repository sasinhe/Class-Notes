\documentclass[english, a4paper,12pt]{Iart}

\usepackage{amsthm}

\swapnumbers % optional, of course
\newtheorem{thm}{Theorem}[section] % the main one
\newtheorem{lemma}[thm]{Lemma}
% other statement types

% for specifying a name
\theoremstyle{plain} % just in case the style had changed
\newcommand{\thistheoremname}{}
\newtheorem{genericthm}[thm]{\thistheoremname}
\newenvironment{namedthm}[1]
{\renewcommand{\thistheoremname}{#1}%
	\begin{genericthm}}
	{\end{genericthm}}

\newtheorem{definition}{Definition}

\newtheorem*{genericthm*}{\thistheoremname}
\newenvironment{namedthm*}[1]
{\renewcommand{\thistheoremname}{#1}%
	\begin{genericthm*}}
	{\end{genericthm*}}


\theoremstyle{remark}
\newtheorem*{remark}{Remark}

\theoremstyle{definition}
\newtheorem{example}{Example}[section]

\begin{document}
	
\author{Samir Salmen}
\title{Notes on Probability and Statistics}
\date{February 2021}

\maketitle

\part{Probability}

\part{Basic Statistics}
\chapter{Common Distributions}
\section{What is a distribution?}
A probability distribution is a mathematical function that gives the chance of a given event inside the sample space. Formally, one can say: \\ 

\begin{namedthm*}{Probability distribution function}
	Let $(\Omega, \sigma, \mu)$ be a measurable space. Then, a \textbf{pdf} is a function $f$ with the propriety that for a random variable $X \in \Omega$ we have $Pr[X]$ 
\end{namedthm*}

Distributions are a main subject inside statistics. If a given random data distribution is known, then everything about that data is easy to deduce. Therefore, the main object inside statistical analysis is \textbf{learn everything possible about the underlying data distribution}.	

















\part{Useful Techniques on Statistics}
\chapter{Transformations and Expectations}






\end{document}