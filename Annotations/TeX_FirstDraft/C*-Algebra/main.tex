\documentclass[english, a4paper,12pt]{Iart}

\usepackage{amsmath}
\usepackage{amsfonts}
\usepackage{amsthm}
\usepackage{tcolorbox}
\usepackage{mathtools}
\swapnumbers % optional, of course
\newtheorem{thm}{Theorem}[section] % the main one
\newtheorem{lemma}[thm]{Lemma}
% other statement types

% for specifying a name
\theoremstyle{plain} % just in case the style had changed
\newcommand{\thistheoremname}{}
\newtheorem{genericthm}[thm]{\thistheoremname}
\newenvironment{namedthm}[1]
{\renewcommand{\thistheoremname}{#1}%
	\begin{genericthm}}
	{\end{genericthm}}

\newtheorem{definition}{Definition}

\newtheorem*{genericthm*}{\thistheoremname}
\newenvironment{namedthm*}[1]
{\renewcommand{\thistheoremname}{#1}%
	\begin{genericthm*}}	{\end{genericthm*}}


\theoremstyle{remark}
\newtheorem*{remark}{Remark}

\theoremstyle{definition}
\newtheorem{example}{Example}[section]

\newtheorem{theorem}{Theorem}

\begin{document}

\author{Samir Salmen}
\title{C*-Algebra}
\date{February 2021}
\newcommand{\C}{\mathbb{C}}
\newcommand{\R}{\mathbb{R}}

\newcommand{\cgeqp}{\succeq^P}

\newcommand{\cleqp}{\preceq^P}

\newcommand{\defeq}{\vcentcolon=}
\newcommand{\eqdef}{=\vcentcolon}

\maketitle


\chapter{Order vector spaces and positivity}

\section{Basic Notions}
\subsection{Cones}

Let $V$ be a vector space over a field. We say that $P \subset V$ is a \textbf{cone} if: $\forall \lambda \geq 0 \text{ and } v \in P, \lambda v \in P$.\\
A given cone is \textbf{convex} if it is closed under vector addiction.\\
Note that a convex cone is also a convex set.\\
\subsection{Preorders}
Any convex cone $P \subset V$ naturally defines a pre-order relation in V, denoted here by $\cgeqp$:
\begin{align*}
	v \cgeqp v' \iff v-v' \in P,\, v,v' \in  V 
\end{align*}
from that follows:
\begin{itemize}
	\item For all $v,v',w,w'\in P$ with $v \cgeqp v'$ and $w \cgeqp w'$, one has $v+w \cgeqp v'+w'$.
	\item For all $\lambda \geq 0$ and $v, v' \in  P$ with $v \cgeqp v'$, one has $\lambda v \cgeqp \lambda v'$
\end{itemize}
Conversely, if $\succeq$ is a pre-order in V with the above two proprieties, then it tis the preorder associeted with the convex cone
\begin{align*}
	V^+ \defeq  \{v \in V: v \succeq 0\}.
.\end{align*}

\begin{definition}
	The pair $(V, \succeq)$ is called "preordered vector space", while the elements that are greater in this order than zero are called the "positive elements".
\end{definition}







\end{document}



