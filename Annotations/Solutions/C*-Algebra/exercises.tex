\documentclass[
% all of the below options are optional and can be left out
% course name (default: 2IL50 Data Structures)
course = {{C*-Algebra}},
% quartile (default: 3)
quartile = {{1}},
% assignment number/name (default: 1)
assignment = 1,
% student name (default: Some One)
name = {{Samir Salmen}},
% student number, NOT S-number (default: 0123456)
studentnumber = {{NUSP: 11298636}},
% student email (default: s.one@student.tue.nl)
email = {{samir.salmen@usp.br}},
% first exercise number (default: 1)
firstexercise = 1
]{aga-homework}

\newtheorem*{theorem*}{Theorem}
\newtheorem{theorem}{Theorem}
\renewcommand*{\proofname}{Proof}
\begin{document}

\exercise
	\subexercise
	We want to prove that:
	\begin{align*}
		\forall f', f'' (f' \succeq f'') \in  \mathcal{F}_{\tilde{\Omega}}, f-f'' \succeq 0, f'-f \succeq 0,
		\Longrightarrow f(p) = 0 \forall p \in \tilde{\Omega}
	.\end{align*}	
	Note that if for all p in $\tilde{\Omega}$ $f'(p)-f(p) \geq 0$, then note that $f(p) \leq 0$. The opposite can be discovered from the relation with $f''$. Therefore $f(p) = 0 \forall p \in  \tilde{\Omega}$.
	\subexercise
		We want to prove an analogous relation of the above one. Only that our objective is to prove that
		$\lim_{p\longrightarrow \infty} f(p) = 0$. Therefore, note that taking  the limit of the inequality $f(p) -
		f''(p) \geq 0$ (and of the analogous one with $f'$) we conclude that$\lim_{p\longrightarrow \infty} f(p) = 0$.

\exercise
		
\end{document}
