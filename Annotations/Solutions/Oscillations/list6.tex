\documentclass[
% all of the below options are optional and can be left out
% course name (default: 2IL50 Data Structures)
course = {{ITVO}},
% quartile (default: 3)
quartile = {{1}},
% assignment number/name (default: 1)
assignment = 6,
% student name (default: Some One)
name = {{Samir Salmen}},
% student number, NOT S-number (default: 0123456)
studentnumber = {{NUSP: 11298636}},
% student email (default: s.one@student.tue.nl)
email = {{samir.salmen@usp.br}},
% first exercise number (default: 1)
firstexercise = 1
]{aga-homework}

\newtheorem*{theorem*}{Theorem}
\newtheorem{theorem}{Theorem}
\renewcommand*{\proofname}{Proof}

\usepackage{dsfont}
\begin{document}
	
	\exercise
	Our Equation of motion is given by:
	\begin{equation}
		\ddot{\phi} - \frac{g + \ddot{h}}{l}\phi = 0
	\end{equation}
	
	We may also define $a(t) = \frac{g + \ddot{h}}{l}$.
		\subexercise
			Lets define $a(t)$ in the following way:
			\begin{equation}
				a(t) =
				\begin{cases}
					w_1^2, \, t \in (0, \pi)\\
					w_2^2, \, t \in (\pi, 2\pi)
				\end{cases}
			\end{equation}
			
			where in both cases $|\ddot{h}(t)| < g$.\\
			
			
			
			Lets calculate $F_1$. The equation of motion for $[0,\pi]$ is:
			
			\begin{equation}
				\ddot{\phi} + \omega_1^2\phi = 0
			\end{equation}
			
			The general solution for this equation is simply:
			
			\begin{equation}
				\phi(t) = c_1e^{\omega_1t}+ c_2e^{-\omega_1t}
			\end{equation}
			
			Therefore, $\dot{\phi(t)} = \omega_1c_1e^{\omega_1t} - c_2e^{-\omega_1t}$. From that, one may deduce the following:
			
			\begin{align}
				x &= \begin{pmatrix}
					\phi\\
					\dot{\phi}
				\end{pmatrix}
				= 		
				\begin{pmatrix}
					e^{\omega_1t} & e^{-\omega_1t}\\
					\omega_1e^{\omega_1t} & -\omega_1e^{-\omega_1t}
				\end{pmatrix} 
				\begin{pmatrix}
					c_1\\
					c_2
				\end{pmatrix}
			\end{align}
			
			In order to discover the columns of $F_1$, lets define some initial conditions.\\
			For $F_1$, lets fix an initial condition $
			x(0) = \begin{pmatrix}
				1\\
				0
			\end{pmatrix}	
			$.
			This implies:
			\begin{equation}
				\begin{cases}
					c_1 + c_2 = 1\\
					c_1 \omega_1 - c_2 \omega_1 = 0
				\end{cases}
			\end{equation}
			Therefore, $a_1 = a_2 = \frac{1}{2}$ in this case.
			
			Analogously, for the second column of $F_1$ lets define $x(0) = \begin{pmatrix}
				0\\
				1
			\end{pmatrix}$. This implies that $c_1 = -c_2 = \frac{1}{\omega_1}$.\\
			This same analysis can be done for $F_2$. Therefore, $F_1$ and $F_2$ are given as follows:
			
			\begin{equation}
				F_1 = \begin{pmatrix}\cosh(\omega_1\pi) & \frac{1}{\omega_1}\sinh(\omega_1\pi)\\ \omega_1\sinh(\omega_1\pi) &  \cosh(\omega_1\pi) \end{pmatrix}
			\end{equation}
			\begin{equation}
				F_2 = \begin{pmatrix}\cosh(\omega_2\pi) & \frac{1}{\omega_2}\sinh(\omega_2\pi)\\ \omega_2\sinh(\omega_2\pi) &  \cosh(\omega_2\pi) \end{pmatrix}
			\end{equation}
			
			We know that $F = F_2F_1$. Therefore (Assuming $\omega_1 < \omega_2)$:
			
			\begin{align}
				|\text{tr}(F)| &= |2\cosh(\omega_1\pi)\cosh(\omega_2\pi) + (\frac{\omega_2}{\omega_1} + \frac{\omega_1}{\omega_2})\sinh(\omega_1\pi)\sinh(\omega_2\pi)|\\
				|\text{tr}(F)| &< |2\cosh(\omega_1\pi)\cosh(\omega_2\pi) + \sinh(\omega_1\pi)\sinh(\omega_2\pi)|\\
				|\text{tr}(F)| &< \left|\cosh([\omega_1 + \omega_2]\pi)\right| < \left|\cosh(\pi\sqrt{\frac{4g}{l}})\right|
			\end{align}
			
			Notice now that if $l < \frac{4g\pi^2}{[\cosh^{-1}(2)]^2} \approx 227.6 \, m$, then $|\text{tr}(F)| < 2$. Therefore the system is stable except for very long handles.
			
		\subexercise
			This problem is analogous to the previous one, except for some things. First in the period from $[\pi, 2\pi]$ we know that $a(t) = -\omega_2^2 < 0$, second we know that $|\ddot{h}(t)| > g, \, \forall t$. Therefore, $F_1$ and $F_2$ are:
			\begin{equation}
				F_1 = \begin{pmatrix}\cosh(\omega_1\pi) & \frac{1}{\omega_1}\sinh(\omega_1\pi)\\ \omega_1\sinh(\omega_1\pi) &  \cosh(\omega_1\pi) \end{pmatrix}
			\end{equation}
		
			$F_2$ deduction is slightly different this time, the solution of the original ODE (when studied inside $[\pi, 2\pi]$) gives the following:
			\begin{equation}
				\ddot{\phi} - \omega_1^2\phi = 0
			\end{equation}
			
			The solution for this equation is:
			\begin{equation}
				\phi(t) = c_1\sin(\omega_2 t) + c_2\cos(\omega_2 t)
			\end{equation}
			
			Therefore, 
			
			\begin{align}
				x &= \begin{pmatrix}
					\phi\\
					\dot{\phi}
				\end{pmatrix}
				= 		
				\begin{pmatrix}
					\sin(\omega_2 t) & \cos(\omega_2 t)\\
					\omega_2\cos(\omega_2 t) & -\omega_2\sin(\omega_2 t)
				\end{pmatrix} 
				\begin{pmatrix}
					c_3\\
					c_4
				\end{pmatrix}
			\end{align}
			
		
			
			setting 
			$
			x(0) = \begin{pmatrix}
				1\\
				0
			\end{pmatrix}	
			$ 
			gives $c_3 = 0$, $c_4 = 1$. The other initial condition used on the first problem gives $c_3 = \frac{1}{\omega_2}$, $c_4 = 0$. Therefore, we know that:
			
			\begin{equation}
				F_2 = \begin{pmatrix}\cos(\omega_2\pi) & \frac{1}{\omega_2}\sin(\omega_2\pi)\\ -\omega_2\sin(\omega_=2\pi) &  \cos(\omega_2\pi) \end{pmatrix}
			\end{equation}
			
			Notice that F contains products of hyperbolic functions with trigonometric ones. Therefore, a stability analysis without numerical values for each constant is improbable.
			
			
			
			
			
			
			
			
	\exercise
		\subexercise
			Our Lagrangian for each mass is given by $\mathcal{L} = T - U$, where $T_{2i+1} = \frac{m_1}{2}\dot{x}_{2i+1}$, $T_{2i} = \frac{m_2}{2}\dot{x}_{2i}$ and $U_i = -\frac{-k}{2}(2x_i - x_{i+1} - x_{i-1})^2$. Therefore, by Euler-Lagrange equation:
			
			
			\begin{align}
				\begin{cases}
					m_1\ddot{x_n} + k(2x_n - y_{n-1} - y_{n}) &= 0\\
					m_2\ddot{y_n} + k(2y_n - x_{n} - x_{n+1}) &= 0
				\end{cases}
			\end{align}
			
			Where we corrected the indices to $n \in \mathbb{Z}$ in order to stay consistent with the statement of the problem.\\
		\subexercise
			First, lets assume a solution of the form:
			
			\begin{equation}
				\begin{pmatrix}x_n\\y_n\end{pmatrix} =\begin{pmatrix}\zeta(t)\\\eta(t)\end{pmatrix} e^{ikn}
			\end{equation}
			
			Therefore, we have through the equation of motion:
			
			\begin{align}
				\begin{cases}	
					m_1\ddot{\zeta}(t)e^{ikn} + k(2\zeta(t)e^{ikn} - \eta(t)e^{ik(n-1)} - \eta(t)e^{ikn}) = 0\\
					m_2\ddot{\eta}(t)e^{ikn} + k(2\eta(t)e^{ikn} - \zeta(t)e^{ikn} - \zeta(t)e^{ik(n+1)}) = 0
				\end{cases}	
			\end{align}
			
			Dividing everything by $e^{ikn}$ we have:
			
			\begin{align}
				\begin{cases}
						m_1\ddot{\zeta}(t) + 2k\zeta(t)- k(e^{-ik} + 1)\eta(t) = 0\\
						m_2\ddot{\eta}(t) + 2k\eta(t) - k(e^{ik} + 1)\zeta(t) = 0
				\end{cases}
			\end{align}
			
			In a matrix form, we have $\left(\text{where} \, \vec{q} = \begin{pmatrix}\zeta(t)\\\eta(t)\end{pmatrix}\right)$:
			
			\begin{equation}
				\ddot{\vec{q}} + \underbrace{\begin{pmatrix}2\omega_1^2 & -\omega_1^2(e^{-ik} + 1)\\ -\omega_2^2(e^{ik} + 1) & 2\omega_2^2\end{pmatrix}}_\text{K}\vec{q} = 0
			\end{equation}
			or alternatively:
			\begin{equation}
				\ddot{\vec{q}} + K\vec{q} = 0
			\end{equation}
			
			Therefore, in order to find the general solution for this ODE, one need only to find the eigenvalue $\lambda$ that solves:
			\begin{equation}
				\det[\lambda^2\mathds{1} + K] = 0
			\end{equation}
				
			\begin{align}
				&\det\left[\lambda^2 \begin{pmatrix}1 & 0\\ 0 & 1\end{pmatrix}+ \begin{pmatrix}2\omega_1^2 & -\omega_1^2(e^{-ik} + 1)\\ -\omega_2^2(e^{ik} + 1) & 2\omega_2^2\end{pmatrix} \right] = 0\\
				&\det\left[\begin{pmatrix}2\omega_1^2+ \lambda^2 & -\omega_1^2(e^{-ik} + 1)\\ -\omega_2^2(e^{ik} + 1) & 2\omega_2^2 + \lambda^2\end{pmatrix}\right] = 0\\
				&(\lambda^2+2\omega_1^2)	(\lambda^2+2\omega_2^2) - \omega_1^2\omega_2^2(e^{-ik}+1)(e^{ik}+1) = 0\\
			\end{align}
			Thus, the eigenvalues of this matrix are:
			\begin{align}
				\begin{cases}
					\lambda = \pm \sqrt{-\omega_1^2-\omega_2^2 \pm \sqrt{\omega_1^4+\omega_2^4 + 2\omega_1^2\omega_2^2\cos(k)}}
				\end{cases}
			\end{align}
		
			After calculating the eigenvectors $\vec{u}_i$ of each eigenvalue $\lambda_i$, one may find the general solution as:
			
			\begin{equation}
				\vec{q}(t) = \sum_{i=1}^{4}e^{\lambda_i t}\vec{u}_i
			\end{equation}
			
			
		
		
\end{document}