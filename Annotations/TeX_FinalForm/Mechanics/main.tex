\documentclass[english, a4paper,12pt]{Iart}

\usepackage{amsthm}
\usepackage{amsmath}

\swapnumbers % optional, of course
\newtheorem{thm}{Theorem}[section] % the main one
\newtheorem{lemma}[thm]{Lemma}
% other statement types

% for specifying a name
\theoremstyle{plain} % just in case the style had changed
\newcommand{\thistheoremname}{}
\newtheorem{genericthm}[thm]{\thistheoremname}
\newenvironment{namedthm}[1]
{\renewcommand{\thistheoremname}{#1}%
	\begin{genericthm}}
	{\end{genericthm}}

\newtheorem{definition}{Definition}

\newtheorem*{genericthm*}{\thistheoremname}
\newenvironment{namedthm*}[1]
{\renewcommand{\thistheoremname}{#1}%
	\begin{genericthm*}}
	{\end{genericthm*}}


\theoremstyle{remark}
\newtheorem*{remark}{Remark}

\theoremstyle{definition}
\newtheorem{example}{Example}[section]

\begin{document}

\author{Samir Salmen}
\title{Notes on Classical Mechanics}
\date{February 2021}



%Ideas:
%Add a "What will you learn in this chapter box"





\maketitle

	\chapter*{Preface}
		% TODO
		\TODO{add some interesting citation here}
	
	\chapter*{Recommended Bibliography}
		Beside this notes, the reader is encouraged to seek other sources, among which there are some classics (and other more obscure textbooks):
		\smallskip
	
		
		\begin{itemize}
			\item H. Goldstein, C. Poole and J. Safko, \textit{Classical Mechanics}\\
			It is the main reference of this notes and of almost every first course on Classical Mechanics, a must-read for everyone interesting in studying the subject.\\
			
			% TODO
			\TODO{add some comments about how Goldstein aproaches the problem}
			
			\item Arnold, A. Weinstein, K. Vogtmann -
			\textit{Mathematical Methods Of Classical Mechanics}
		
		\end{itemize}
		
	\chapter*{What is this book about}


\part{Newtonian Mechanics}
	\chapter{Kinematics and a introduction to Mechanics}
		We'll first start studying the simplest system possible, a infinitesimal point moving through empty space. But before that, we must first define some concepts.
		
		\section{Key concepts to Kinematics}
			\subsection{Vectors and Scalars}
	
							


	\chapter{Newtonian Mechanics in 1-D}
		
	
	
\part{Lagrangian and Hamiltonian Formulations}
	\chapter{Elementary Principles}
		In this chapter, we'll do a quick review on some concepts from the previous part while also adding	some key notions for understanding other formulations of mechanics.
		\section{Mechanics of a Particle}
			Let $\vec{r}$ be the position of an given particle to the origin. We know that the
			velocity vector $\vec{v}$ is given by:
			\begin{align*}
				\vec{v} =\frac{d\vec{r}}{dx} 
			.\end{align*}
			The linear momentum of the particle is defined as the product of the velocity with
			it's mass:
			\begin{align*}
				\vec{p} = m\vec{v}
			.\end{align*}		
			This equations alone can't describe more interesting systems, because they don't
			account for forces external to the particle. Therefore, we need some relation between
			the linear momentum and the forces acting on the mass. This is given by
			\textbf{Newton's Second Law}:
			\begin{align*}
				\sum_{n=1}^{N}\vec{F}_n = \frac{d\vec{p}}{dx}
			.\end{align*}
			This equation of motion is a diferential equation (which can be tranformed to a
			diferential equation of second order with respect to $\vec{r}$, assuming that
			$\vec{F}$ does not depend on higher order
			derivatives)\\
			
			A reference frame in which Newton's Law is valid is called a inertial frame. And one
			which it isn't is called a non-inertial frame.\\
			Note that for diferentiating inertial" to non-inertial ones we need \textbf{Newton's Third
			Law}. \textit{Only by detecting forces without an action-reaction pair} one may detect that they are on a 
			non-inertial frame.\\
			




		\section{Conservation Laws}

		\subsection{Mechanics of a System of Particles}
		
		\section{Degrees of Freedom}
		We are motivated to define a dynamical variable
		as any set of variables that \textit{fully}
		describe our system in its totality (including
		the change of it undet the action of the forces).
		We obviously also want to it be a minimal set.
		Therefore, what we are looking are some variables
		that are fully indepentent between them (and
		are themselves functions of time). This functions
		can be found by solving Newton's Law together
		with some initial conditions corresponding to
		each variable.\\
		Usually the dynamic variables are chosen to be
		positions and/or angles. For a free particle on
		3D space for example, the dynamic variables are the 
		cartesian coordinates $\vec{r}(t) = (x(t), y(t),
		z(t))$.\\
		Unspecified dynamical variables $q_k$ are
		referred to as \textit{generalized coordinates}.
		Note that the number of degrees of freedom is an
		intrisic proprierty of the system, however what
		are the actual coordinates is up to us. This have
		to do withe the vector structure of the phase
		space, more on that later.\\
		\section{Constraints}
		\textit{Constraints} reduce the number of degrees
		of freedom. Originally, a $N$ dimensional space
		with $M$ particles have $N*M$ degrees of freedom.
		This is a lot of variables to take track of, so
		the concept of constrains help us to minimize
		this number.\\
		Take for example a 2D pendullum with radius $R$.
		Even thou it is on 2D space,it only takes a single
		angle $\theta$ to fully describe the system.\\
		Therefore if there are $j$ \textit{indepentent}
		constrains, the system has only $V = N*M - j$
		degrees of freedom.\\
		In general, we have two different types of
		degrees of freedom. The \textit{explicit time dependence}
		constrains are the ones we call
		\textit{holonomic}, while time indepentent
		constrains are called \textit{scleronomic}.
		
	\chapter{Lagrangian Formalism}
		
		At this point, one may ask: "Why do we need other formulations for Classical Mechanics? Isn't Newton's formulation complete? (in a sense that it can describe all physical systems)".\\
		%todo double pendullum example.
		The answer for the first question is highlighted by example 	
		(???), It shows that albeit Newton's formulation is indeed complete,when deriving the equations of motion starting from the diagram of forces it cannot account correctly for one thing, \textbf{Human mistake}.
		
		
		
		The \textit{Action} S is defined as:
		\begin{equation}
			S = \int_{t_0}^{t_1} \mathcal{L}(q, \dot{q}, t) \, dt
		\end{equation}
		
		The Principle of Least Action states as follows.
		
		\begin{namedthm*}{Principle of Least Action}
			The evolution of a physical system in the time interval $[t_0, t_1]$ corresponds to a stationary point of the action.
		\end{namedthm*}
		
		Even thou the action usually takes a minimum on the true trajectory of the system, this isn't always true, as \href{http://www.eftaylor.com/pub/Gray&TaylorAJP.pdf}{this example shows}.\\
		
		Before we go any further, I would like to make a brief discussion about the intuition behind the Least Action principle.
		
		%todo
		\TODO{add discurssion about Least action principle}
	
	
	
	\chapter{Exoteric formulations of Classical Mechanics}
			
	

	
\end{document}
